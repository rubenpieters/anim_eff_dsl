\section{Comparison to Sequential/Callback style}
\label{sec:evaluation}

This section gives a qualitative evaluation of \dsl{} to GreenSock Animation Platform (GSAP). This is a widely recommended JavaScript animation library, which we believe is representative for many other animation libraries.

GSAP is a mature library, and as a result presents am extensive amount of features to its users. Some examples are relative property changes, animation primitives related to scalable vector graphics (SVG) images, portability across browsers, various easing functions, \ldots

The goal of this section is not to compare feature lists of both, since the maturity of GSAP makes it the clear winner. Instead, this section compares the foundational blocks on which the approaches are built. For the GSAP library, the main building block is \emph{sequential} composition, and only a very limited form of parallel composition. However, \emph{callbacks} are provided for various points within an animation which can be used to simulate parallel elements. Thus, we call this the \textbf{sequential/callback} approach. This is in contrast with the \textbf{sequential/parallel} approach of \dsl{}, where \emph{parallel} composition is a foundational element.

\subsection{Simulating Parallel}

TODO: actually define a \texttt{parallel} function

In this section we discuss how the \hs{parallel} combinator from \dsl{} can be simulated in GSAP.

The equivalent of \hs{basic} is the creation of a \texttt{TweenMax} object. The arguments are also similar:  a parameter which indicates which object to change, a duration, and the target value for the property. Below we create the animation \texttt{right} which moves \texttt{box1} to the right.

\begin{js}
const right = new TweenMax("#box1", 1, {x: 50});
\end{js}

Constructing a sequential animation is done by using a timeline. First a timeline is created and we can add animations with the \texttt{add} method. Below, we create the animation \texttt{rightThenDown} which moves \texttt{box1} to the right and then moves it down.

\begin{js}
const rightThenDown = new TimelineMax()
  .add(new TweenMax("#box1", 1, {x: 50}))
  .add(new TweenMax("#box1", 1, {y: 50}));
\end{js}

To create an animation which moves both \texttt{box1} and \texttt{box2} in parallel to the right, we do this by supplying the same optional position parameter. If we provide both \texttt{to} calls with the same static position \texttt{0}, then both animations run in parallel. Below we create a \texttt{both} animation which moves both \texttt{box1} and \texttt{box2} in parallel.

\begin{js}
const both = new TimelineMax()
  .add(new TweenMax("#box1", 1, {x: 50}), 0)
  .add(new TweenMax("#box2", 1, {x: 50}), 0);
\end{js}

However, embedding these parallel animations into larger ones is more difficult since we need to provide the exact starting points for all animations which need to run in parallel. We discuss two ways of doing this: calculating offsets and using callbacks.

\subsubsection{Offsets}

We can create a custom sequential function which calculates the duration of the first animation and provides this as offset for the second animation. GSAP conveniently provides a method to calculate the duration of an animation which we can use for this.

\begin{js}
function sequential(tl1, tl2) {
  return new TimelineMax()
    .add(tl1.play())
    .add(tl2.play(), tl1.totalDuration());
}
\end{js}

\subsubsection{Callbacks}

A different approach to simulate the \hs{parallel} combinator is to use the \texttt{onComplete} callback of the timeline. We register a callback on the first animation to play the second animation, and then return the reference to the first animation.

\begin{js}
function sequential(tl1, tl2) {
  tl1.eventCallback("onComplete", () => tl2.play());
  return tl1;
}
\end{js}

\subsubsection{Parallel Animations}

We can then use both of these approaches follows to create an animation where both boxes move to the right in parallel and then move back in parallel.

\begin{js}
const both = new TimelineMax({ paused: true })
  .add(new TweenMax("#box1", 1, {x: 50}), 0)
  .add(new TweenMax("#box2", 1, {x: 50}), 0);

const moveBackBox1 = new TimelineMax({ paused: true })
  .add(new TweenMax("#box1", 1, {x: 0}), 0)
  .add(new TweenMax("#box2", 1, {x: 0}), 0);

const bothThenBack = sequential(both, moveBackBox1).play();
\end{js}

\subsection{The Problem}

So far everything seems to work just fine, we were able to create parallel animations as in \dsl{} on top of the functionality provided by GSAP. The problems lie in the interaction between the provided duration feature and callbacks, which makes each of them unreliable in presence of the other.

As an example, when we calculate the duration of the \texttt{bothThenBack} defined with the offset approach we get the correct result \texttt{2}. However, if we calculate the duration of the \texttt{bothThenBack} animation defined in the callback approach we get \texttt{1}. The duration calculation does not (and can not) take into account animations activated in callbacks. Which also means that the offset approach will stop working correctly in the presence of animations created with callbacks.

Limiting ourselves to never using callbacks does not seem an option either, since it is the functionality which provides the expressivity aspect of the library. For example, if we want to embed a new effect inside an animation which GSAP has not provided for us, then we have to use the callback functionality to implement this. And as a result the duration calculation and parallel composition will break again.

Back in Haskell terminology, GSAP is presenting itself as an \hs{Applicative} DSL for inspection purposes, but a \hs{Monadic} DSL for expressing animations. And rather than throwing an error upon inspecting the duration of a monadic animation, it gives a result which is not necessarily what the user might expect. This is why we argue that the sequential/parallel foundation is an improvement over the sequential/callback foundation.
