\section{Extra Features}
\label{sec:features}

In addition to the previous features, we want to support some additional ones: extending with custom operations, inspecting animations, and extending with custom combinators.

TODO: motivation of features by comparing with GSAP and refer to that section

\subsection{Extensibility}
\label{sec:customop}

The \hs{basic} operation and the \hs{sequential} and \hs{parallel} combinators form the basis for expressing a variety of animations. However, there are many different situations which require different primitives to express our desired animation.

An example in our use case is the \hs{checkIcon} animation, which is part of the \hs{markAsDone} animation, where it is required to set the color of the checkmark to a specific value. We can define an additional \hs{set} operation and embed it inside a \dsl{} animation.

\begin{spec}
checkIcon = do
  basic (checkmark . scale) (For 0.05) (To 0)
  set (checkmark . color) completeGreen
  (basic (circle . extra) (For 0.2) (To 360))
    `parallel`
    (basic (checkmark . scale) (For 0.2) (To 1.2))
  basic (checkmark . scale) (For 0.05) (To 1)
\end{spec}

\begin{figure}[h]
\centering
\includegraphics[width=\figscale\textwidth]{pictures/todo}
\caption{The \hs{checkIcon} animation.}
\label{fig:}
\end{figure}

\subsection{Inspectability}

\dsl{} is inspectable, meaning that we can derive properties of expressed computations by \emph{inspecting them rather than running them}. For example, we might want to know the duration of \hs{menuAnimation} without actually running it and keeping track of the time. We can do this by using a predefined \hs{duration} function, which calculates the duration by inspecting the animation. This gives a duration of \hs{0.5} seconds, which is indeed the duration of two \hs{0.5} second animations in parallel.

\begin{spec}
menuAnimDuration = duration menuAnimation
-- menuAnimDuration = 0.5 
\end{spec}

Of course, it is not possible to inspect every animation. In the following situation we have a custom operation \hs{getValue} which returns a \hs{Float}. If the result of this value is used as the duration parameter, then we cannot know upfront how long the animation will last. Requesting to calculate the duration of this animation results in a type error.

\begin{spec}
complicatedAnimation = do
  v <- getValue
  basic lens (For v) (To 10)

compAnimDuration = duration complicatedAnimation
-- type error
\end{spec}

\subsection{Expressiveness}
\label{sec:customcomb}

Similarly to providing custom operations, \dsl{} also supports custom combinators. For typical DSLs this is not a requirement since working with monadic computations provides the combinators \hs{>>=} and \hs{return} which are suitable for the required use cases. However, since \dsl{} has the additional requirement of being inspectable, the \hs{>>=} combinator can end up being a liability because it only provide a very limited amount of inspectability.

In the \hs{onlyDone} animation, we show all done todo items, while hiding all not done items. This could be implemented by first showing all items with the \hs{showAll} animation, since an item might have been hidden by a previous action, and then hiding all not done items with the \hs{hideNotDone} animation. The definition for this is given below.

\begin{spec}
onlyDoneNaive = do showAll ; hideNotDone
\end{spec}

However, this animation is a bit naive since it executes the \hs{showAll} animation regardless of whether there are any hidden done items that actually need to appear. Instead, we first check whether there are any done items and based on that we play the naive version of \hs{onlyDone}, otherwise we just hide the not done items.

\begin{spec}
onlyDone = do
  cond <- doneItemsGt0
  if cond
    then onlyDoneNaive
    else hideNotDone
\end{spec}

However, if we also want to inspect this computation the previous formulation is problematic. Instead, we can define a custom combinator \hs{ifThenElse} which captures the form of animation we want to express.

\begin{spec}
onlyDone = ifThenElse doneItemsGt0
  onlyDoneNaive
  hideNotDone
\end{spec}

For this new combinator, we can define custom ways to inspect it. For example, we might want to retrieve the \emph{maximum} duration of an animation. Since retrieving the actual duration is not sensible as both branches might have different durations, which is in fact the case.

\begin{spec}
onlyDoneMaxDuration = maxDuration onlyDone
-- onlyDoneMaxDuration = 1
\end{spec}

These sections gave a taste of the features of \dsl{} and what using it feels like. In the following sections, we delve deeper into the internals of the implementation.
