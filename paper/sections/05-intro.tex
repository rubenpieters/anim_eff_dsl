\section{Introduction}
\label{sec:intro}

Monads have become ubiquitous since they were made popular in functional programming \cite{DBLP:conf/lfp/Wadler90}. However, while monads are useful for structuring effectful code in a pure functional codebase, they do carry certain drawbacks. First, monads are not trivially extensible. A variety of techniques have been developed to resolve this, including monad transformers \cite{DBLP:conf/popl/LiangHJ95}, free monads, and algebraic effects and handlers \cite{DBLP:conf/esop/PlotkinP09}. Second, monadic computations can only be inspected up to the next action. Techniques such as applicative functors \cite{DBLP:journals/jfp/McbrideP08}, arrows \cite{DBLP:journals/scp/Hughes00}, or selective applicative functors increase the inspection capabilities by reducing the expressivity of our computations.

This paper develops a domain specific language (DSL), called \dsl{}, for micro-animations embbeded inside Haskell. \dsl{} employs these techniques to support its key features: extensibility of operations, inspectability of animations while still being as expressive as needed.

Micro-animations, or animated transitions, are a commonly used concept within user experience (UX) design for software applications. When used appropriately, they aid the user in understanding evolving states of the application \cite{DBLP:conf/infovis/BedersonB99} \cite{DBLP:conf/chi/Gonzalez96} \cite{DBLP:journals/tvcg/HeerR07}. Examples can be seen in various applications: window managers shrink minimized windows and move them towards the taskbar, menus in mobile applications pop in gradually, browsers highlight newly selected tabs with an animation.

