\section{Introduction}
\label{sec:intro}

Monads quickly became ubiquitous in functional programming because of their ability to structure effectful code in a pure functional codebase \cite{DBLP:conf/lfp/Wadler90}. They have since seen wide use in DSLs for Haskell. However, there are certain trade-offs when choosing for monadic power in DSLs. The obvious advantage is their expressivity, but there are also some drawbacks to monadic DSLs. The main loss is that of \emph{inspectability}, since monadic computations can only be inspected up to the next action. Techniques such as applicative functors \cite{DBLP:journals/jfp/McbrideP08}, arrows \cite{DBLP:journals/scp/Hughes00}, or selective applicative functors \cite{Mokhov:2019:SAF:3352468.3341694} choose the other side of the trade-off: they increase the inspection capabilities by reducing the expressivity compared to monads.

This paper develops a domain specific language (DSL) embedded in Haskell for defining micro-animations, called \dsl{}\footnote{Pronounced \textit{pace}, the name is derived from \textit{\textbf{pa}rallel} and \textit{\textbf{se}quential}.}. \dsl{} employs a technique which alleviates some aspects of the trade-off between expressivity and inspectability. The expressivity of control flow is restricted by means of type classes, similar to the MTL-style. When the animations in \dsl{} have restricted forms of control flow, such as applicative animations, then class instances can be provided to support inspectability. The MTL-style is also an open encoding which allows extensions to the syntax of the DSL. The limited expressivity can be increased by extending the control flow constructs of the DSL while inspectability is preserved if a corresponding instance is implemented.

Micro-animations are short animations displayed when users interact with an application, for example an animated transition between two screens. When used appropriately, they aid the user in understanding evolving states of the application \cite{DBLP:conf/infovis/BedersonB99,DBLP:conf/chi/Gonzalez96,DBLP:journals/tvcg/HeerR07}. Examples can be found in almost every software application: window managers shrink minimized windows and move them towards the taskbar, menus in mobile applications pop in gradually, browsers highlight newly selected tabs with an animation, etc.

\dsl{} combines various features expected of animation libraries, by building them on top of recent ideas found in functional programming. More concretely, our contributions are as follows:
\begin{itemize}
\item We develop \dsl{}, which supports correct interactions for expressing arbitrary animations and inspectability. Animation libraries, such as the GreenSock Animation Platform (GSAP)\footnote{\url{https://greensock.com}}, typically use callbacks as a means of extensibility/expressivity which is detrimental to inspectability. We show an example where this results in unexpected behaviour, and how to correct it.
\item We extend and develop \dsl{} as an example of an \emph{extensible} DSL, in the sense that operations can be freely added. While there is a variety of literature on approaches to extensibility, there is not much literature devoted to use cases.
\item Animations in \dsl{} can be specified in an \emph{inspectable} manner, in the sense that information is obtained from a computation without running it. While inspectability has been studied for specific types of computations, such as free applicative functors \cite{DBLP:journals/corr/CapriottiK14}, it is a novel aspect to combine it with extensibility.
\item \dsl{} contains a primitive for parallel animation composition. This composition form can be arbitrarily nested and embedded anywhere within an animation and interacts correctly with the other features of \dsl{}: extensibility and inspectability. Parallel components in sequentially specified components is not new, see for example the \emph{par} element in UML sequence diagrams \cite{umlspec}, or the parallel statement in Ren'Py\footnote{\url{https://www.renpy.org/doc/html/atl.html\#parallel-statement}} or React Native Animations\footnote{\url{https://facebook.github.io/react-native/docs/animated\#parallel}}. But, those systems either do not provide a form of parallelism as general or interaction with other features such as inspectability is not present.
\item We have implemented various example applications with \dsl{}: a to-do list application presented in this paper, a communication story example, a game-like demo application and a pacman game\footnote{\url{https://github.com/rubenpieters/PaSe-hs/tree/master/PaSe-examples}}. PaSe is agnostic with respect to its GUI backend, some of the applications were built using \texttt{gloss}\footnote{\url{https://hackage.haskell.org/package/gloss}}, while others were built using the Haskell SDL bindings\footnote{\url{https://hackage.haskell.org/package/sdl2}}. The pacman application was developed both in Haskell with PaSe and in TypeScript with GSAP. This paper discusses a quantitative comparison of the development of both applications.
\end{itemize}
