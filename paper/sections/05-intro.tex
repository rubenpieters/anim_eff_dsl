\section{Introduction}
\label{sec:intro}

Monads are ubiquitous ever since they were made popular in functional programming \cite{DBLP:conf/lfp/Wadler90}. However, while monads are useful for structuring effectful code in a pure functional codebase, they do carry certain drawbacks. First, monads are not trivially composable. A variety of techniques have been developed to resolve this, including monad transformers \cite{DBLP:conf/popl/LiangHJ95}, free monads, and algebraic effects and handlers \cite{DBLP:conf/esop/PlotkinP09}. Second, monadic computations can only be inspected up to the next action. Again, we are able to use previously developed techniques such as applicative functors \cite{DBLP:journals/jfp/McbrideP08}, arrows \cite{DBLP:journals/scp/Hughes00}, or selective applicative functors to increase the inspection capabilities by reducing the expressivity of our computations.

This paper addresses the question whether it is possible to handle both concerns at once. We do this by looking at the use case of developing a domain specific language (DSL) for graphical user interface (GUI) animations. This use case requires that the DSL is simultaneously as expressive as needed, as inspectable as needed, and extendable with new operations. These requirements are not problematic when it is known upfront which expressiveness is needed. But our DSL itself does not limit the expressiveness and the programmer is free to choose the amount of inspectability vs expressiveness for the expressed computations.
