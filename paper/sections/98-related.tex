\section{Related Work}
\label{sec:related}

\paragraph{Functional Reactive Programming}
The origins of functional reactive programming (FRP) lie in the creation of
animations \cite{DBLP:conf/icfp/ElliottH97}, and many later developments use
FRP as the basis for purely functional GUIs. 

\dsl{} focuses on easily describing \emph{micro-animations}, which differ from general
\emph{animoations} as considered by FRP. The latter can typically be described
by a time-paramterized picture function \hs{Time -> Picture}. While a subset of
all possible animations, micro-animations are not easily described by such a
function because many small micro-animations can be active at the same time and
their timing depends on user interaction.

We have only supplied an implementation of \dsl{} on top of a traditional
event-based framework, but it is interesting future work to investigate an
implementation of the \hs{linearTo}, \hs{sequential} and \hs{parallel}
operations in terms of FRP behaviours and events.

\paragraph{Animation Frameworks}

Typical micro-animation libraries for web applications (with CSS or JavaScript)
and animation constructions in game engines provide a variety of configurable
pre-made operations while composing complex animations or integrating new types
of operations is difficult. \dsl{} focuses on the creation of complex sequences
of events while still providing the ability to embed new animation primitives.
We have looked at GSAP as an example of such libraries and some of the limits
in combining extensibility with callbacks and inspectability. \dsl{} is an
exercise in improving this combination of features forward in a direction which
is more predictable for the user.

\paragraph{Planning-Based Animations}

\dsl{} shares similarities with approaches which specify an animation as a plan
which needs to be executed
\cite{DBLP:conf/chi/KurlanderL95,DBLP:conf/eics/MirlacherPB12}. An
animation is specified by a series of steps to be executed, the
plan of the animation. The coordinator, which manages and advances the
animations, is implemented as part of the hosting application. \dsl{} realizes
these plan-based animations with only a few core principles and features the
possibility of adding custom operations and inspection. A detailed comparison
with these approaches is difficult, since their works are very light on
details of the actual implementation aspect.

\paragraph{Inspectable DSLs}

Some DSLs for
parsing~\cite{DBLP:journals/scp/Hughes00,DBLP:journals/corr/CapriottiK14,DBLP:conf/icfp/Lindley14},
non-determinism~\cite{DBLP:journals/corr/abs-1905-06544}, remote
execution~\cite{DBLP:conf/haskell/Gibbons16,DBLP:conf/haskell/GillSDEFGRSS15}
and build systems~\cite{DBLP:journals/pacmpl/MokhovMJ18} focus on inspectability aspects, yet
none of them provide extensibility and expressiveness in addition to inspection.

