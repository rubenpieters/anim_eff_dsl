\section{Interaction Between Expressiveness and Inspectability}
\label{sec:interaction}

A Haskell DSL is typically a monadic one due to the expressive power of the \hs{>>=} combinator. However, this power also hinders the inspectability of the DSL. In this section we take a look at balancing the wanted expressiveness vs inspectability by introducing custom combinators.

\subsection{Branching}

Recall the \hs{rareAnimation} definition from Section~\ref{sec:customcomb}, a definition using \hs{do}-notation is given below. Now it contains a \hs{Monad} constraint on \hs{f} which means that the animation is not inspectable.

\begin{code}
rareAnimationDo :: (Monad f, Basic RareWorld f, Rng f) => f ()
rareAnimationDo = do
  x <- rng (1, 10)
  if x == 10
    then specialAnimation
    else normalAnimation
\end{code}

However, it does seem like we should be able to extract some duration related information from it. For example, the maximum duration should be the highest of the duration of \hs{specialAnimation} and \hs{normalAnimation} --- and similarly for the minimum duration.

To express this idea using the DSL we need to introduce an explicit combinator for the constructions which piggyback on the \hs{>>=} cominator. In this case this is the \hs{if-then-else} construction, for which we create a new custom combinator. This combinator is a different formulation of selective applicative functors \cite{Mokhov:2019:SAF:3352468.3341694}.

\begin{code}
class IfThenElse f where
  ifThenElse :: f Bool -> f a -> f a -> f a
\end{code}

In the instance for \hs{Animation}, we can use the implementation dependent on the \hs{>>=} combinator. So, we retrieve the value inside \hs{f}~\hs{Bool} and decide on this value whether to continue with the \texttt{then} branch or the \texttt{else} branch.

\begin{code}
instance (Monad f) => IfThenElse (Animation obj f) where
  ifThenElse fBool thenBranch elseBranch = do
    bool <- fBool
    if bool then thenBranch else elseBranch
\end{code}

To retrieve the maximum duration, we use a different \hs{newtype} definition, \hs{MaxDuration}, to signify that we are not simply calculating the duration of the animation. In the instance definition we retrieve the durations of the \texttt{then} and \texttt{else} branches and add the higher value to the duration of the preceding animation inside the condition.

\begin{code}
 instance IfThenElse (Const MaxDuration) where
   ifThenElse (Const (MaxDur durCondition))
              (Const (MaxDur durThen))
              (Const (MaxDur durElse)) =
   Const (MaxDur (durCondition + max durThen durElse))
\end{code}

\subsection{Data Dependency}


