\section{Interaction Between Expressiveness and Inspectability}
\label{sec:interaction}

A Haskell DSL is typically a monadic one due to the expressive power of the \hs{>>=} combinator. However, this power also hinders the inspectability of \dsl{}. In this section we take a look at balancing the wanted expressiveness vs inspectability by introducing custom combinators.

\subsection{Branching}

Recall the \hs{rareAnimation} definition from Section~\ref{sec:customcomb}, a definition using \hs{do}-notation is given below. Now it contains a \hs{Monad} constraint on \hs{f} which means that the animation is not inspectable.

\begin{code}
rareAnimationDo :: (Monad f, Basic RareWorld f, Rng f) => f ()
rareAnimationDo = do
  x <- rng (1, 10)
  if x == 10
    then specialAnimation
    else normalAnimation
\end{code}

However, it does seem like we should be able to extract some duration related information from it. For example, the maximum duration should be the highest of the duration of \hs{specialAnimation} and \hs{normalAnimation} --- and similarly for the minimum duration.

To express this idea using \dsl{} we need to introduce an explicit combinator for the constructions which piggyback on the \hs{>>=} cominator. In this case this is the \hs{if-then-else} construction, for which we create a new custom combinator. This is the \hs{handle} combinator from the \hs{DynamicIdiom} class \cite{DBLP:phd/ethos/Yallop10} and the \hs{ifS} combinator from the \hs{Selective} class \cite{Mokhov:2019:SAF:3352468.3341694}.

\begin{code}
class IfThenElse f where
  ifThenElse :: f Bool -> f a -> f a -> f a
\end{code}

In the instance for \hs{Animation}, we can use the implementation dependent on the \hs{>>=} combinator. So, we retrieve the value inside \hs{f}~\hs{Bool} and decide on this value whether to continue with the \texttt{then} branch or the \texttt{else} branch.

\begin{code}
instance (Monad f) => IfThenElse (Animation obj f) where
  ifThenElse fBool thenBranch elseBranch = do
    bool <- fBool
    if bool then thenBranch else elseBranch
\end{code}

To retrieve the maximum duration, we use a different \hs{newtype} definition, \hs{MaxDuration}, to signify that we are not simply calculating the duration of the animation. In the instance definition we retrieve the durations of the \texttt{then} and \texttt{else} branches and add the higher value to the duration of the preceding animation inside the condition.

\begin{code}
 instance IfThenElse (Const MaxDuration) where
   ifThenElse (Const (MaxDur durCondition))
              (Const (MaxDur durThen))
              (Const (MaxDur durElse)) =
   Const (MaxDur (durCondition + max durThen durElse))
\end{code}

\subsection{Data Dependency}

The \hs{mouseBoxAnimation} from Section~\ref{sec:customop} defines a monadic computation. Notice that it has data flow dependencies: the particle identifier \hs{i} is used later as the input for the \hs{basic} and \hs{delete} operations. However, this data flow should not impact the calculation of a duration for our animation, since the duration is independent from this identifier.

We need to indicate in our operations how the data dependency flows, to show which values are (or are not) dependent on which other values. Simply adding a new combinator or operation is not sufficient for solving this problem. The arrow abstraction \cite{DBLP:journals/scp/Hughes00} was designed for these use cases. For animations which require this detailed data dependency analysis, we switch to an arrow version of \dsl{}. The arrow version of \hs{mouseBoxAnimation} is given below. This does require some boilerplate to support all operations in the arrow version.

\begin{code}
mouseBoxAnimationA ::
  (ParticleA MouseWorld a,
   BasicA MouseWorld a, ParA a, Arrow a) =>
   a (Float, Float) ()
mouseBoxAnimationA = proc (mouseX, mouseY) -> do
  i <- createA -< (createBoxParticle mouseX mouseY)
  parA (arr (\i -> MyLens (msParticles . withId i . scale))
        >>> basicA (For 0.5) (To 4))
       (arr (\i -> MyLens (msParticles . withId i . scale))
        >>> basicA (For 0.5) (To 4)) -< i
  deleteA -< (deleteBoxParticle, i)
\end{code}

From this representation of the animation it is possible to calculate the duration.

\begin{code}
mouseBoxAnimDuration :: Duration
mouseBoxAnimDuration = durationA mouseBoxAnimationA
\end{code}
