\section{Introduction}
\label{sec:intro}


Because of their ability to structure effectful code in a pure functional codebase, monads quickly became ubiquitous in functional programming~\cite{DBLP:conf/lfp/Wadler90}. They have since seen wide use in Haskell Domain Specific Languages (DSLs). However, the choice for a monadic DSL implies certain trade-offs.  The obvious advantage of monadic DSLs is their expressivity, but there are also drawbacks. The main loss is that of \emph{inspectability}, as monadic computations can only be inspected up to the next action.  Techniques such as applicative functors \cite{DBLP:journals/jfp/McbrideP08}, arrows \cite{DBLP:journals/scp/Hughes00}, or selective applicative functors \cite{Mokhov:2019:SAF:3352468.3341694} choose the other side of the trade-off: they increase the inspection capabilities by reducing the expressivity compared to monads.

This paper develops a DSL embedded in Haskell for defining micro-animations, called \dsl{}\footnote{Pronounced \textit{pace} (\textipa{pe\textsci{}s}), the name is derived from \textit{\textbf{Pa}rallel} and \textit{\textbf{Se}quential}.}. \dsl{} employs a technique which alleviates some aspects of the trade-off between expressivity and inspectability. The expressivity of control flow is restricted by means of type classes, inspired by the MTL style originally introduced by Liang~\emph{et al.}~\cite{DBLP:conf/popl/LiangHJ95}. The MTL style is an open encoding which allows extensions to the syntax of the DSL. Instantiating the abstract animation definitions with, for example, the \hs{Const} functor provides inspectability.  Expressivity can be increased, while preserving inspectability, by adding new control flow constructs to the DSL and providing a corresponding instance for inspection.

Micro-animations are short animations displayed when users interact with an application, for example an animated transition between two screens. When used appropriately, they aid the user in understanding evolving states of the application \cite{DBLP:conf/infovis/BedersonB99,DBLP:conf/chi/Gonzalez96,DBLP:journals/tvcg/HeerR07}.  Examples can be found in almost every software application: window managers animate window minimization, menus in mobile applications pop in gradually, browsers highlight newly selected tabs with an animation, etc.

\dsl{} provides the features expected of animation libraries by building them with recent ideas from functional programming. Our contributions are as follows:
\begin{itemize}
\item We develop \dsl{}, which supports arbitrary composition of animations and inspectability. Animation libraries, such as the GreenSock Animation Platform (GSAP)\footnote{\url{https://greensock.com}}, typically use callbacks as a means of extensibility/expressivity; this is detrimental to inspectability. We show an example resulting in unexpected behaviour and how \dsl{} correctly handles it.
\item \dsl{} is an \emph{extensible} DSL: the syntax can be extended with new operations. The animations use case is novel for approaches to extensibility.
\item \dsl{} supports \emph{inspectability}: extracting information from computations before running it. Inspectability is present in specific computation classes, such as free applicatives \cite{DBLP:journals/corr/CapriottiK14}. But, it is novel to combine it with extensibility.
\item \dsl{} supports arbitrary nesting of parallel and sequential animations which correctly interacts with inspectability. Such parallel components exist already, see for example Ren'Py\footnote{\url{https://www.renpy.org/doc/html/atl.html\#parallel-statement}}, React Native Animations\footnote{\url{https://facebook.github.io/react-native/docs/animated\#parallel}} or Qt Animations\footnote{\url{https://doc.qt.io/qt-5/animation.html}}. Yet, general-purpose animation libraries lack them. Also, we correctly support the interaction with inspectability.
\item We implemented various examples\footnote{\url{https://github.com/rubenpieters/PaSe-hs/tree/master/PaSe-examples}}: a to-do list application, a communication story example, a game-like demo application and a Pac-Man game. We combined \dsl{} with both \texttt{gloss}\footnote{\url{https://hackage.haskell.org/package/gloss}} and the Haskell SDL bindings\footnote{\url{https://hackage.haskell.org/package/sdl2}} as graphics backend. This paper uses the to-do list as motivating application and compares the development of the Pac-Man application, developed in both Haskell with PaSe and in TypeScript with GSAP.
\end{itemize}
